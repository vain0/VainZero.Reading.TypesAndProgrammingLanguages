\part{型無しの計算体系}
\chapter{型無し算術式}
\section{導入}

\section{構文}

\begin{jexercise*}[3.2.4]
  略
\end{jexercise*}
\begin{jproof}
  任意の $i \in \bbN$ について、$m = |S_i|$ とおくと、定義から $|S_{i+1}| = m^3 + 3m + 3$ となる。
  \begin{align*}
    |S_1| & = 3,
    \\ |S_2| & = 3^3 + 3 \cdot 3 + 3 = 39,
    \\ |S_3| & = 39^3 + 3 \cdot 39 + 3 = 59439.
  \end{align*}
\end{jproof}

\begin{jexercise*}[3.2.5.]
  略
\end{jexercise*}
\begin{jproof}
  任意の $i \in \bbN$ をとる。
  数学的帰納法により、$S_i \subset S_{i+1}$ を示す。
  $i = 0$ のとき、$S_0 = \emptyset \subset S_1$ である。
  $i > 0$ とする。
  帰納法の仮定より、$S_{i-1} \subset S_i$ である。
  任意の項 $t \in S_i$ をとる。
  $t \in \set{\true, \false, 0}$ のとき、$t \in S_{i+1}$ は明らか。
  $t \in \setc{\operatorname{succ} s}{s \in S_{i-1}}$ のとき、$s \in S_{i-1} \subset S_i$ であるから、$t \in \setc{\operatorname{succ} s}{s \in S_i} \subset S_{i+1}$ となる。
  他のケースも同様。
  \qed
\end{jproof}
