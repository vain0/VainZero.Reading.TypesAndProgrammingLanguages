\part{型無しの計算体系}
\chapter{型無し算術式}
\section{導入}

\section{構文}

\begin{jexercise*}[3.2.4]
  略
\end{jexercise*}
\begin{jproof}
  任意の $i \in \bbN$ について、$m = |S_i|$ とおくと、定義から $|S_{i+1}| = m^3 + 3m + 3$ となる。
  \begin{align*}
    |S_1| & = 3,
    \\ |S_2| & = 3^3 + 3 \cdot 3 + 3 = 39,
    \\ |S_3| & = 39^3 + 3 \cdot 39 + 3 = 59439.
  \end{align*}
  \qed
\end{jproof}
