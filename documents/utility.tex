% \newcommand など。

\theoremstyle{definition}
\newtheorem{jtheorem}{定理}[section]
\newtheorem*{jtheorem*}{定理}
\newtheorem{jdefinition}[jtheorem]{定義}
\newtheorem*{jdefinition*}{定義}
\newtheorem{jlemma}[jtheorem]{補題}
\newtheorem*{jlemma*}{補題}
\newtheorem*{jprop*}{命題}
\newtheorem*{jexercise*}{演習}
\newtheorem*{jremark*}{注意}

% \newcommand{\diff}{\mathrm{d}}

% proof
\makeatletter
\newenvironment{jproof}[1][\proofHeader]{\par
  \normalfont
  \topsep6\p@\@plus6\p@ \trivlist
  \item[\hskip\labelsep{\bfseries #1}\@addpunct{\bfseries.}]\ignorespaces
}{\endtrivlist}
\newcommand{\proofHeader}{証明}
\makeatother

% Logic.

\newcommand{\imply}{\Rightarrow}
\newcommand{\equivalent}{\Leftrightarrow}

% Sets.

\newcommand{\bbN}{\mathbb{N}}
\newcommand{\bbZ}{\mathbb{Z}}
\newcommand{\bbQ}{\mathbb{Q}}
\newcommand{\bbR}{\mathbb{R}}
\newcommand\set[1]{\{#1\}}
\newcommand\setc[2]{\{\,#1 \mid #2\,\}}
\newcommand\card[1]{\operatorname{card}(#1)}
\newcommand\xrange[2]{[#1, #2)}

% Domain-specific.

\newcommand{\op}[1]{\operatorname{#1}}
\def\code#1{\texttt{#1}}

\newcommand{\true}{\operatorname{true}}
\newcommand{\false}{\operatorname{false}}
\newcommand{\depth}{\operatorname{depth}}
\newcommand{\then}{\operatorname{then}}
\newcommand{\IfThenElse}[3]{
  \operatorname{if} #1
  \operatorname{then} #2
  \operatorname{else} #3
}
\newcommand{\wrong}{\operatorname{wrong}}
