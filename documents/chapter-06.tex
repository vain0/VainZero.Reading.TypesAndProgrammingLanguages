\chapter{項の名無し表現}

\section{項と文脈}

\begin{jexercise*}[6.1.1]
  コンビネーターを名無し項に変換せよ。
\end{jexercise*}
\begin{itembox}[l]{解答}
  \code{$c_0$ = $\lambda$. $\lambda$. 0}

  \code{$c_2$ = $\lambda$. $\lambda$. 1 (1 0)}

  \code{plus = $\lambda$. $\lambda$. $\lambda$. $\lambda$. 3 1 (2 1 0)}

  \code{fix = $\lambda$. ($\lambda$. 1 ($\lambda$. (1 1) 0)) ($\lambda$. 1 ($\lambda$. (1 1) 0))}

  \code{foo = ($\lambda$. ($\lambda$. 0)) ($\lambda$. 0)}
\end{itembox}

\section{シフトと代入}

\begin{jexercise*}[6.2.2]
  シフトされた項を計算せよ。
\end{jexercise*}
\begin{itembox}[l]{解答}
  \code{
    $\uparrow^2$($\lambda$. $\lambda$. 1 (0 2))
    = $\lambda$. $\lambda$. 1 (0 4)
  }

  \code{
    $\uparrow^2$($\lambda$. 0 1 ($\lambda$. 0 1 2))
    = $\lambda$. 0 3 ($\lambda$. 0 1 4)
  }
\end{itembox}
