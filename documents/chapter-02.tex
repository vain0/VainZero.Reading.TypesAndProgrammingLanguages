\chapter{数学的準備}

\section{集合、関係、関数}

\section{順序集合}

\begin{jexercise*}[2.2.6]
  略
\end{jexercise*}
\begin{jproof}
  $R$ を含む任意の反射的関係 $S$ をとる。
  任意の $(s, t) \in R'$ をとる。
  和集合の定義から、
    (a) $(s, t) \in R$ または
    (b) $(s, t) \in \setc{(s, s)}{s \in S}$
    である。
  \begin{description}
    \item[(a) のとき]
      $R \subset S$ より $(s, t) \in S$ となる。
    \item[(b) のとき]
      $S$ は反射的であるから、 $(s, t) = (s, s) \in S$ となる。
  \end{description}
  よって $R' \subset S$ となる。
  したがって、$R'$ は $R$ を含む最小の反射的関係である。
  すなわち、$R'$ は $R$ の反射的閉包といえる。
  \qed
\end{jproof}

\begin{jexercise*}[2.2.7]
  略
\end{jexercise*}
\begin{jremark*}
  このことから、$R$ の推移的閉包は常に存在する。
  また、推移的閉包の最小性から、それは一意である。
  したがって、$R^+$ という記法には曖昧性がないことが分かる。
\end{jremark*}
\begin{jproof}
  $R$ を含む任意の推移的関係 $S$ をとる。
  $R^+ \subset S$ を示せば十分である。
  任意の $n \in \bbN$ をとる。
  数学的帰納法により、$R_n \subset S$ を示す。
  $n = 0$ のとき、$R_0 = R \subset S$ である。
  次に $n > 0$ とする。
  任意の $(s, u) \in R_n$ をとる。
  もし $(s, u) \in R_{n-1}$ なら、帰納法の仮定より $R_{n-1} \subset S$ なので、$(s, u) \in S$ となる。
  そうでなければ、$R_n$ の定義より、$(s, t) \in R_{n - 1}$ かつ $(t, u) \in R_{n - 1}$ となる $t$ がある。
  帰納法の仮定より、$R_{n - 1} \subset S$ である。
  $S$ は推移的なので、$(s, u) \in S$ がいえる。
  \qed
\end{jproof}

\begin{jtheorem}[反射的推移的閉包の構成的定義]
  \label{反射的推移的閉包の構成的定義}
  $R$ を関係とする。
  $R' = R \cup \setc{(s, s)}{s \in R}$ とおく。
  $R^* = (R')^+$ は $R$ の反射的推移的閉包である。
\end{jtheorem}
\begin{jproof}
  $R$ を含む任意の反射的推移的関係 $S$ をとる。
  $R^* \subset S$ を示せば十分。
  $(R')^+$ を演習 2.2.7 と同様に、列 $(R'_i \mid i \in \bbN)$ の和集合として定義する。
  任意の $n \in \bbN$ をとる。
  数学的帰納法により、$R'_n \subset S$ を示す。これを示せば十分である。

  $n = 0$ のとき、演習 2.2.6 より $R'$ は $R$ の反射的閉包であるから、$R$ を含む反射的関係 $S$ に含まれる。すなわち、$R'_0 = R' \subset S$。

  $n > 0$ とする。
  任意の $(s, u) \in R'_n$ とおく。
  もし $(s, u) \in R'_{n-1}$ なら、帰納法の仮定より $R'_{n-1} \subset S$ なので、$(s, u) \in S$ となる。
  そうでなければ、$R'_n$ の定義より、ある $t$ で、$(s, t) \in R'_{n-1}$ かつ $(t, u) \in R'_{n-1}$ となるものが存在する。
  帰納法の仮定より、$R'_{n-1} \subset S$ である。
  $S$ は推移的なので、$(s, u) \in S$ がいえる。
  \qed
\end{jproof}

\begin{jdefinition}[推移列]
  $S$ を集合とし、$R$ を $S$ 上の2項関係とする。
  $S$ に含まれる長さ $n \in \bbN$ の列 $(s_i)_{i=0}^{n-1}$ が次の条件を満たすとき、これを $R$ の推移列と呼ぶ:
  任意の $i \in \xrange{0}{n-1}$ について、$(s_i, s_{i+1}) \in R$。
\end{jdefinition}
\begin{jremark*}
  要するに $s_0 R s_1 R \cdots R s_{n-1}$ という列のこと。
  これは $S$ を点集合、$R$ を辺集合とする有向グラフ $G = (S, R)$ における walk である。
\end{jremark*}

\begin{jtheorem}[反射的推移的閉包の推移列]
  \label{反射的推移的閉包の推移列}
  $R$ を集合 $S$ 上の2項関係とする。
  任意の $(s, t) \in R^*$ について、$R$ の推移列 $(s_i)_{i=0}^{m-1}$ で、$s_0 = s$ かつ $s_{m-1} = t$ となるものが存在する。
\end{jtheorem}
\begin{jproof}
  $R^*$ を定理 \ref{反射的推移的閉包の構成的定義} と同様に和集合で定義する。
  任意の $n \in \bbN$ をとる。
  数学的帰納法により、「任意の $(s, t) \in R'_n$ について、$R$ の推移列 $(s_i)_{i=0}^{m-1}$ で、$s_0 = s$ かつ $s_{m-1} = t$ となるものが存在する」ことを示す。

  $n = 0$ のとき、任意の $(s, t) \in R'_0 = R'$ をとる。
  $s = t$ の場合は、長さ 1 の列 $(s)$ が推移列となる。
  $(s, t) \in R$ の場合は、長さ 2 の $(s, t)$ が推移列となる。

  $n > 0$ とする。
  任意の $(s, u) \in R'_n$ をとる。
  $(s, u) \in R'_{n-1}$ の場合は、数学的帰納法の仮定から求める推移列を得られる。
  そうでない場合は、ある $t$ で、$(s, t) \in R'_{n-1}$ かつ $(t, u) \in R'_{n-1}$ となるものが存在する。
  数学的帰納法の仮定より、$s$ から $t$ への推移列 $(s_i)_{i=0}^{m-1}$ と、$t$ から $u$ への推移列 $(t_j)_{j=0}^{l-1}$ がある。
  この2つの列を $s_{m-1} = t_0$ で接合することにより、$s$ から $u$ への長さ $m + l - 1$ の推移列を得る。

  最後に、任意の $(s, t) \in R^*$ をとる。
  ある $n$ について $(s, t) \in R'_n$ なので、数学的帰納法の帰結から、$s$ から $t$ への推移列が存在する。
  \qed
\end{jproof}

\begin{jexercise*}[2.2.8]
  略
\end{jexercise*}
\begin{jproof}
  任意の $(s, t) \in R^*$ をとる。
  $P(s)$ とする。
  定理 \ref{反射的推移的閉包の推移列} より、$s$ から $t$ への $R$ の推移列 $(s_i)_{i=0}^{n-1}$ がある。

  任意の $i \in \bbN$ をとる。$i < n$ とする。
  数学的帰納法により、$P(s_i)$ を示す。
  $i = 0$ のとき、$s_0 = s$ なので $P(s_0)$ が成り立つ。
  $i > 0$ のとき、数学的帰納法の仮定より $P(s_{i-1})$ がいえる。
  推移列の定義より $(s_{i-1}, s_i) \in R$ である。
  $P$ は $R$ によって保存されるので、 $P(s_i)$ がいえる。

  特に $i = n - 1$ のとき、$P(s_{n-1})$ つまり $P(t)$ がいえる。
  \qed
\end{jproof}

\section{列}

\section{帰納法}
