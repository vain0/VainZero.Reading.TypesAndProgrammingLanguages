\chapter{数学的準備}

\begin{jexercise*}[2.2.6]
  略
\end{jexercise*}
\begin{jproof}
  $R$ を含む任意の反射的関係 $S$ をとる。
  任意の $(s, t) \in R'$ をとる。
  和集合の定義から、
    (a) $(s, t) \in R$ または
    (b) $(s, t) \in \setc{(s, s)}{s \in S}$
    である。
  \begin{description}
    \item[(a) のとき]
      $R \subset S$ より $(s, t) \in S$ となる。
    \item[(b) のとき]
      $S$ は反射的であるから、 $(s, t) = (s, s) \in S$ となる。
  \end{description}
  よって $R' \subset S$ となる。
  したがって、$R'$ は $R$ を含む最小の反射的関係である。
  すなわち、$R'$ は $R$ の反射的閉包といえる。
  \qed
\end{jproof}

\begin{jexercise*}[2.2.7]
  略
\end{jexercise*}
\begin{jproof}
  $R$ の推移的閉包 ($R$ を含む最小の推移的関係) を $S$ とおく。

  まず $R^+ \subset S$ を示す。
  任意の $n \in \bbN$ をとる。
  数学的帰納法により、$R_n \subset S$ を示す。
  $n = 0$ のとき、$R_0 = R \subset S$ である。
  次に $n > 0$ とする。
  任意の $(s, u) \in R_n$ をとる。
  $R_n$ の定義より、$(s, t) \in R_{n - 1}$ かつ $(t, u) \in R_{n - 1}$ となる $t$ がある。
  帰納法の仮定より、$R_{n - 1} \subset S$ である。
  $S$ は推移的なので、$(s, u) \in S$ がいえる。

  次に $S \subset R^+$ を示す。
  $R^+$ が推移的関係であることを示せば十分である。
  任意の $(s, t) \in R^+$ と $(t, u) \in R^+$ をとる。
  $R^+$ の定義より、$(s, t) \in R_i$ となる $i$ があり、
  同時に、$(t, u) \in R_j$ となる $j$ がある。
  $k = \max(i, j)$ とおく。
  $R_i \subset R_k$ かつ $R_j \subset R_k$ である。
  よって $(s, t) \in R_k$ かつ $(t, u) \in R_k$ となるから、
  $(s, u) \in R_{k + 1} \subset R^+$ 。
  \qed
\end{jproof}
