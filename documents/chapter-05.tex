\chapter{型無しラムダ計算}

\section{基礎}

\section{ラムダ計算でのプログラミング}

\begin{jdefinition}
  \begin{align*}
    \true & = \lambda x y. x
    \\ \false & = \lambda x y. y
  \end{align*}
\end{jdefinition}

真偽値がif式の機能を持つのが興味深い。
唐突に意識を高めると、「ラムダ項による定義はその値の本質的な操作を表している」といえそうだ。

\begin{jexercise*}[5.2.1]
  or, not を表すラムダ抽象を定義せよ。
\end{jexercise*}
\begin{itembox}[l]{解答}
  \begin{itemize}
    \item
      $\operatorname{or}(l, r) = \IfThenElse{l}{\true}{r}$ であるから、
      $\operatorname{or} = \lambda l r. l \true r$
      と定義すればよい。
    \item
      $\operatorname{not} = \lambda x. x \false \true$
  \end{itemize}
\end{itembox}

\begin{align*}
  \operatorname{pair} & = \lambda x y p. p x y
  \\ \operatorname{fst} & = \lambda t. t \true
  \\ \operatorname{snd} & = \lambda t. t \false
\end{align*}

ペアの構築子は「要素と射影を受け取る関数」として定められる。
射影関数はそのまま、要素の列から1つの値を選択する関数である。
$\true$、$\false$ はペアの射影関数にもなっている。(偶然?)

\begin{align*}
  c_0 & = \lambda f x. x
  \\ c_1 & = \lambda f x. f x
  \\ c_2 & = \lambda f x. f (f x)
  \\ c_{i+1} & = \lambda f x. f (c_i f x)
  \\ \operatorname{succ} & = \lambda n f x. f (n f x)
\end{align*}

Church自然数は関数の冪乗として定義される。

\begin{jexercise*}[5.2.2]
  Chuch数の後者関数 $\operatorname{succ}$ を別の方法で定義せよ。
\end{jexercise*}
\begin{itembox}[l]{解答}
  \begin{align*}
    \operatorname{succ} = \lambda n f x. n f (f x)
  \end{align*}
\end{itembox}
