\chapter{型無しラムダ計算}

\section{基礎}

\section{ラムダ計算でのプログラミング}

\begin{jdefinition}
  \begin{align*}
    \true & = \lambda x y. x
    \\ \false & = \lambda x y. y
  \end{align*}
\end{jdefinition}

真偽値がif式の機能を持つのが興味深い。
唐突に意識を高めると、「ラムダ項による定義はその値の本質的な操作を表している」といえそうだ。

\begin{jexercise*}[5.2.1]
  or, not を表すラムダ抽象を定義せよ。
\end{jexercise*}
\begin{itembox}[l]{解答}
  \begin{itemize}
    \item
      $\operatorname{or}(l, r) = \IfThenElse{l}{\true}{r}$ であるから、
      $\operatorname{or} = \lambda l r. l \true r$
      と定義すればよい。
    \item
      $\operatorname{not} = \lambda x. x \false \true$
  \end{itemize}
\end{itembox}

\begin{align*}
  \operatorname{pair} & = \lambda x y p. p x y
  \\ \operatorname{fst} & = \lambda t. t \true
  \\ \operatorname{snd} & = \lambda t. t \false
\end{align*}

ペアの構築子は「要素と射影を受け取る関数」として定められる。
射影関数はそのまま、要素の列から1つの値を選択する関数である。
$\true$、$\false$ はペアの射影関数にもなっている。(偶然?)

\begin{align*}
  c_0 & = \lambda f x. x
  \\ c_1 & = \lambda f x. f x
  \\ c_2 & = \lambda f x. f (f x)
  \\ c_{i+1} & = \lambda f x. f (c_i f x)
  \\ \operatorname{succ} & = \lambda n f x. f (n f x)
\end{align*}

Church自然数は関数の冪乗として定義される。

\begin{jexercise*}[5.2.2]
  Chuch数の後者関数 $\operatorname{succ}$ を別の方法で定義せよ。
\end{jexercise*}
\begin{itembox}[l]{解答}
  \begin{align*}
    \operatorname{succ} = \lambda n f x. n f (f x)
  \end{align*}
\end{itembox}

\begin{jexercise*}[5.2.3]
  $\operatorname{plus}$ を用いずにChurch数の乗算を定義できるか。
\end{jexercise*}
\begin{itembox}[l]{解答}
  少なくとも私には思いつかない。
\end{itembox}

\begin{jexercise*}[3.2.4]
  $\operatorname{power} = \lambda m n. n (\operatorname{times} m) 1$
\end{jexercise*}

\begin{jexercise*}[5.2.5]
  $\op{monus} = \lambda m n. n \op{pred} m$
\end{jexercise*}

\begin{jexercise*}[5.2.6]
  自然数 $n$ について、\code{pred n} を正規形まで評価するときの計算量を求めよ。
\end{jexercise*}
\begin{itembox}[l]{解答}
  概算でよいとのことなので、\code{snd (pair x y)} $\to$ {y} のような演算は1ステップと数える。

  $n$ の前者 ($n - 1$ または $0$) を $m$ とおく。
  \code{n ss zz} $\to^*$ \code{pair m n} かつ、その評価のステップ数が $O(n)$ であることは、数学的帰納法により示せる。
  実際、任意の $n \in \bbN$ をとる。
  $n = 0$ のとき、
    \code{n ss zz}
    = \code{0 ss zz}
    $\to$ \code{zz}
    = \code{pair 0 0}
  より、1ステップで正規形に評価できる。
  次に $n > 0$ とする。
  $m = n - 1$ とおく。
  $m$ の前者を $l$ とおく。
  \begin{align*}
    \code{n ss zz} & = \code{(succ m) ss zz}
    \\ & \to \code{ss (m ss zz)}
    \\ & \to^* \code{ss (pair l m)}
    \\ & \to \code{pair (snd (pair l m)) (succ (snd (pair l m)))}
    \\ & \to \code{pair m (succ (snd (pair l m)))}
    \\ & \to \code{pair m (succ m)}
    \\ & = \code{pair m n}
  \end{align*}
  となる。
  ステップ数は、帰納法の仮定より $\to^*$ の部分が $O(n)$ であるから、全体も $O(n)$ である。

  \code{pred n = fst (n ss zz)} より、ステップ数は $O(n)$ である。
\end{itembox}

\begin{jexercise*}[5.2.7]
  \code{equal} を定義せよ。
\end{jexercise*}
\begin{itembox}[l]{解答}
  $\operatorname{equal}(n, m) \iff n \leq m \land n \geq m$ であるから、$\leq$ を表す項を定義すればよい。
  $n \leq m \iff \operatorname{monus}(n, m) = 0$ であるから、これは \code{iszero} と演習 5.2.5. の \code{monus} を組み合わせて簡単に定義できる。
\end{itembox}

\begin{jexercise*}[5.2.8]
  Church リストを定義せよ。
\end{jexercise*}
\begin{itembox}[l]{解答}
  \begin{align*}
    \text{\code{nil}}
      & = \text{\code{$\lambda$ f z. z}}
    \\ \text{\code{cons}}
      & = \text{\code{$\lambda$ x xs f z. f x (xs f z)}}
    \\ \text{\code{isnil}}
      & = \text{\code{$\lambda$ xs. xs ($\lambda$ \_ \_. false) true}}
    \\ \text{\code{head}}
      & = \text{\code{$\lambda$ xs. xs ($\lambda$ x \_. some x) none}}
    \\ \text{\code{tail}}
      & = ...
  \end{align*}
\end{itembox}

Church 自然数 \code{n} に \code{succ} を適用した項は、\code{m} ($= n + 1$) を表す Church 自然数そのものではなく、それと {\bf 振る舞い等価な} ラムダ項にしかならない。これは評価戦略により、ラムダ項の内側が簡約されないため。

正規形を持たない (どれだけ簡約しても正規形に到達しない) 項は、{\bf 発散} するという。\code{omega} は明らかに発散する。

\[
  \text{\code{omega = ($\lambda$ x. x x) ($\lambda$ x. x x)}}
\]

\begin{jexercise*}[5.2.9]
  \code{if} を使わずに階乗関数 \code{factorial} を定義せよ。
\end{jexercise*}
\begin{itembox}[l]{解答}
  \code{
    factorial =
      fix ($\lambda$ factorial n.
        ((eq n 0)
          ($\lambda$ \_. 1)
          ($\lambda$ \_. times n (factorial (pred n)))
        ) unit
      )
  }
\end{itembox}

\begin{jexercise*}[5.2.10]
  プリミティブな自然数を Church 自然数に変換する関数 \code{church-nat} を定義せよ。
\end{jexercise*}
\begin{itembox}[l]{解答}
  \code{
    church-nat =
      fix ($\lambda$ church-nat n.
        if iszero n then
          church-0
        else
          church-succ (church-nat (pred n))
      )
  }
\end{itembox}
