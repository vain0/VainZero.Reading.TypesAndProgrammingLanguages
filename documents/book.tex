\documentclass[a4paper, 12pt, oneside, openany]{jsbook}
%
\usepackage
  { amsmath
  , amssymb
  , amsthm
  , bm
  , makeidx
  , verbatim
  , wrapfig
  , fancybox % インライン要素を枠で囲むのに使う。
  , ascmac % ブロック要素を枠で囲むのに使う。
  , bcprules % 推論規則や証明図を書くのに使う。
  , listings % lstlisting 環境を導入する。コードを掲載するのに使う。
  }

\makeindex

\setlength{\textwidth}{\fullwidth}
\setlength{\textheight}{40\baselineskip}
\addtolength{\textheight}{\topskip}
\setlength{\voffset}{-0.55in}

% \newcommand など。

\theoremstyle{definition}
\newtheorem{jtheorem}{定理}[section]
\newtheorem*{jtheorem*}{定理}
\newtheorem{jdefinition}[jtheorem]{定義}
\newtheorem*{jdefinition*}{定義}
\newtheorem{jlemma}[jtheorem]{補題}
\newtheorem*{jlemma*}{補題}
\newtheorem*{jprop*}{命題}
\newtheorem*{jexercise*}{演習}
\newtheorem*{jremark*}{注意}

% \newcommand{\diff}{\mathrm{d}}

% proof
\makeatletter
\newenvironment{jproof}[1][\proofHeader]{\par
  \normalfont
  \topsep6\p@\@plus6\p@ \trivlist
  \item[\hskip\labelsep{\bfseries #1}\@addpunct{\bfseries.}]\ignorespaces
}{\endtrivlist}
\newcommand{\proofHeader}{証明}
\makeatother

% Sets.

\newcommand{\bbN}{\mathbb{N}}
\newcommand{\bbZ}{\mathbb{Z}}
\newcommand{\bbQ}{\mathbb{Q}}
\newcommand{\bbR}{\mathbb{R}}
\newcommand\set[1]{\{#1\}}
\newcommand\setc[2]{\{\,#1 \mid #2\,\}}
\newcommand\card[1]{\operatorname{card}(#1)}
\newcommand\xrange[2]{[#1, #2)}


\title{型システム入門(TaPL) 読書ノート}
\author{vain0}
\date{\today}
\begin{document}

\maketitle

\frontmatter

\addcontentsline{toc}{chapter}{概要}

『型システム入門 プログラミング言語と型の理論』の読書ノート。証明の細部などの行間や、演習の回答などを記す。

\tableofcontents

\mainmatter

\chapter{はじめに}

\chapter{数学的準備}

\begin{jexercise*}[2.2.6]
  略
\end{jexercise*}
\begin{jproof}
  $R$ を含む任意の反射的関係 $S$ をとる。
  任意の $(s, t) \in R'$ をとる。
  和集合の定義から、
    (a) $(s, t) \in R$ または
    (b) $(s, t) \in \setc{(s, s)}{s \in S}$
    である。
  \begin{description}
    \item[(a) のとき]
      $R \subset S$ より $(s, t) \in S$ となる。
    \item[(b) のとき]
      $S$ は反射的であるから、 $(s, t) = (s, s) \in S$ となる。
  \end{description}
  よって $R' \subset S$ となる。
  したがって、$R'$ は $R$ を含む最小の反射的関係である。
  すなわち、$R'$ は $R$ の反射的閉包といえる。
  \qed
\end{jproof}

\begin{jexercise*}[2.2.7]
  略
\end{jexercise*}
\begin{jproof}
  $R$ を含む任意の推移的関係 $S$ をとる。
  $R^+ \subset S$ を示せば十分である。
  任意の $n \in \bbN$ をとる。
  数学的帰納法により、$R_n \subset S$ を示す。
  $n = 0$ のとき、$R_0 = R \subset S$ である。
  次に $n > 0$ とする。
  任意の $(s, u) \in R_n$ をとる。
  $R_n$ の定義より、$(s, t) \in R_{n - 1}$ かつ $(t, u) \in R_{n - 1}$ となる $t$ がある。
  帰納法の仮定より、$R_{n - 1} \subset S$ である。
  $S$ は推移的なので、$(s, u) \in S$ がいえる。
  \qed
\end{jproof}

\part{型無しの計算体系}
\chapter{型無し算術式}
\section{導入}

\section{構文}

\begin{jexercise*}[3.2.4]
  略
\end{jexercise*}
\begin{jproof}
  任意の $i \in \bbN$ について、$m = |S_i|$ とおくと、定義から $|S_{i+1}| = m^3 + 3m + 3$ となる。
  \begin{align*}
    |S_1| & = 3,
    \\ |S_2| & = 3^3 + 3 \cdot 3 + 3 = 39,
    \\ |S_3| & = 39^3 + 3 \cdot 39 + 3 = 59439.
  \end{align*}
\end{jproof}

\begin{jexercise*}[3.2.5.]
  略
\end{jexercise*}
\begin{jproof}
  任意の $i \in \bbN$ をとる。
  数学的帰納法により、$S_i \subset S_{i+1}$ を示す。
  $i = 0$ のとき、$S_0 = \emptyset \subset S_1$ である。
  $i > 0$ とする。
  帰納法の仮定より、$S_{i-1} \subset S_i$ である。
  任意の項 $t \in S_i$ をとる。
  $t \in \set{\true, \false, 0}$ のとき、$t \in S_{i+1}$ は明らか。
  $t \in \setc{\operatorname{succ} s}{s \in S_{i-1}}$ のとき、$s \in S_{i-1} \subset S_i$ であるから、$t \in \setc{\operatorname{succ} s}{s \in S_i} \subset S_{i+1}$ となる。
  他のケースも同様。
  \qed
\end{jproof}

\begin{jtheorem}[整礎帰納法]
  $S$ 上の2項関係 $\leq$ を整礎 (無限降下列を持たない) とする。
  $P$ を $S$ の元に関する述語とする。
  $S$ の元 $s$ に関する述語 $Q(s)$ を「$t < s$ となる任意の $t$ について $P(t)$ が成り立つ」とおく。(これを帰納法の仮定という。)
  さらに、$Q(s)$ を満たす任意の $s \in S$ について $P(s)$ が成り立つと仮定する。
  このとき、任意の $s$ について $P(s)$ が成り立つ。
\end{jtheorem}
\begin{jproof}
  $s \in S$ から始まる $\leq$ の降下列の最大の長さを $d(s)$ とおく。
  $s$ から始まる降下列には長さ1の列 $(s)$ が存在するので、$d(s)$ の値は常に定義され、$d(s) \geq 1$ である。また、$\leq$ が整礎であることから、$d(s) < \infty$ である。

  $n \in \bbN$ に関する述語 $R(n)$ を「$d(s) \leq n$ となる任意の $s \in S$ について、$P(s)$ が成り立つ」とおく。
  任意の $n \in \bbN$ ($n \geq 1$) をとる。
  数学的帰納法により $R(n)$ を示す。

  まず $n = 1$ とする。
  任意の $s \in S$ をとり、$d(s) \leq 1$ とする。
  $Q(s)$ が成り立つことを示すため、$t < s$ となる任意の $t \in S$ をとる。
  長さ2の降下列 $(s, t)$ が存在することから $d(s) \geq 2$ がいえて、矛盾する。
  爆発律により、$P(t)$ が成り立つ。
  こうして $Q(s)$ がいえた。
  この $s$ について定理の仮定を用いて、$P(s)$ を得る。

  次に $n > 1$ の場合を考える。
  任意の $s \in S$ をとり、$d(s) \leq n$ とする。
  $Q(s)$ を示すため、$t < s$ となる任意の $t \in S$ をとる。
  仮に $d(t) \geq n$ とする。
  $t$ から始まる最長の降下列 $(t_i)_{i=1}^{d(t)}$ が存在する。
  このとき、$s > t$ より、列 $(s, t_1, \cdots, t_{d(t)})$ は $s$ から始まる長さ $d(t) + 1$ の降下列である。
  よって $d(s) \geq d(t) + 1 \geq n + 1 > n$ となり、矛盾を得る。
  したがって $d(t) < n$ である。
  帰納法の仮定より、$R(d(t))$ が成り立つので、$P(t)$ がいえる。
  こうして $Q(s)$ がいえた。
  基底ケースと同様に $P(s)$ を得る。
  数学的帰納法により、任意の $n \in \bbN$ ($n \geq 1$) について $R(n)$ が成り立つ。

  最後に、任意の $s \in S$ をとる。
  任意の $t \in S$ をとり、$t < s$ と仮定する。
  $R(d(s))$ より、$P(s)$ が成り立つ。
  よって $Q(s)$ がいえる。
  定理の仮定より $P(s)$ が成り立つ。
  \qed
\end{jproof}

\begin{jexercise*}[定理3.3.4.の証明]
  深さに関する帰納法、サイズに関する帰納法、構造的帰納法を証明せよ。
\end{jexercise*}
\begin{jproof}
  深さに関する帰納法について。
  項の間の関係 $s \leq_d t$ を $\depth(s) \leq \depth(t)$ とおく。
  $\depth$ の定義から、この関係が整礎であることはすぐ分かる。
  整礎帰納法から結論を得る。

  サイズに関する帰納法についても、深さに関する帰納法と同様に、$\operatorname{size}$ が項の構造に関して単調増加することを利用する。

  構造的帰納法について。
  「直接の部分項」という語が未定義なので、まずその定義を述べる。
  項 $t$ の「直接の部分項」全体からなる集合 $\operatorname{Subterms}(t)$ を、次のように定義する。
  \begin{align*}
    \operatorname{Subterms}(\true) & = \emptyset,
    \\ \operatorname{Subterms}(\false) & = \emptyset,
    \\ \operatorname{Subterms}(0) & = \emptyset,
    \\ \operatorname{Subterms}(\operatorname{succ}(t_1))
      & = \set{t_1},
    \\ \operatorname{Subterms}(\operatorname{pred}(t_1))
      & = \set{t_1},
    \\ \operatorname{Subterms}(\operatorname{iszero}(t_1))
      & = \set{t_1},
    \\ \operatorname{Subterms}(\operatorname{if} t_1 \then t_2 \operatorname{else} t_3)
      & = \set{t_1, t_2, t_3}.
  \end{align*}
  そして「項 $s$ が項 $t$ の直接の部分項である」を $s \in \operatorname{Subterms}(t)$ と定める。

  この関係は整礎である。
  なぜなら、項 $t$ から始まる任意の下降列 $(s_i)_{i=1}^n$ ($n$ は自然数または無限) について、列 $(\depth(s_i))_{i=1}^n$ が下降列であるため。
  したがって、これも深さに関する帰納法と同様に、整礎帰納法に帰着できる。
  \qed
\end{jproof}

\chapter{算術式のML実装}

略

\chapter{型無しラムダ計算}

\section{基礎}

\section{ラムダ計算でのプログラミング}

\begin{jdefinition}
  \begin{align*}
    \true & = \lambda x y. x
    \\ \false & = \lambda x y. y
  \end{align*}
\end{jdefinition}

真偽値がif式の機能を持つのが興味深い。
唐突に意識を高めると、「ラムダ項による定義はその値の本質的な操作を表している」といえそうだ。

\begin{jexercise*}[5.2.1]
  or, not を表すラムダ抽象を定義せよ。
\end{jexercise*}
\begin{itembox}[l]{解答}
  \begin{itemize}
    \item
      $\operatorname{or}(l, r) = \IfThenElse{l}{\true}{r}$ であるから、
      $\operatorname{or} = \lambda l r. l \true r$
      と定義すればよい。
    \item
      $\operatorname{not} = \lambda x. x \false \true$
  \end{itemize}
\end{itembox}

\begin{align*}
  \operatorname{pair} & = \lambda x y p. p x y
  \\ \operatorname{fst} & = \lambda t. t \true
  \\ \operatorname{snd} & = \lambda t. t \false
\end{align*}

ペアの構築子は「要素と射影を受け取る関数」として定められる。
射影関数はそのまま、要素の列から1つの値を選択する関数である。
$\true$、$\false$ はペアの射影関数にもなっている。(偶然?)

\begin{align*}
  c_0 & = \lambda f x. x
  \\ c_1 & = \lambda f x. f x
  \\ c_2 & = \lambda f x. f (f x)
  \\ c_{i+1} & = \lambda f x. f (c_i f x)
  \\ \operatorname{succ} & = \lambda n f x. f (n f x)
\end{align*}

Church自然数は関数の冪乗として定義される。

\begin{jexercise*}[5.2.2]
  Chuch数の後者関数 $\operatorname{succ}$ を別の方法で定義せよ。
\end{jexercise*}
\begin{itembox}[l]{解答}
  \begin{align*}
    \operatorname{succ} = \lambda n f x. n f (f x)
  \end{align*}
\end{itembox}


% \appendix
% \include{appendixA}

\chapter*{謝辞}
\addcontentsline{toc}{chapter}{謝辞}

感謝

% \include{biblography}

\newpage
\printindex

\end{document}
