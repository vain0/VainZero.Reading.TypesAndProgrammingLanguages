\documentclass[a4paper, 12pt, oneside, openany]{jsbook}
%
\usepackage
  { amsmath
  , amssymb
  , amsthm
  , bm
  , makeidx
  , verbatim
  , wrapfig
  , fancybox % 枠で囲むのに便利
  }

\makeindex

\setlength{\textwidth}{\fullwidth}
\setlength{\textheight}{40\baselineskip}
\addtolength{\textheight}{\topskip}
\setlength{\voffset}{-0.55in}

% \newcommand など。

\theoremstyle{definition}
\newtheorem{jtheorem}{定理}[section]
\newtheorem*{jtheorem*}{定理}
\newtheorem{jdefinition}[jtheorem]{定義}
\newtheorem*{jdefinition*}{定義}
\newtheorem{jlemma}[jtheorem]{補題}
\newtheorem*{jlemma*}{補題}
\newtheorem*{jprop*}{命題}
\newtheorem*{jexercise*}{演習}
\newtheorem*{jremark*}{注意}

% \newcommand{\diff}{\mathrm{d}}

% proof
\makeatletter
\newenvironment{jproof}[1][\proofHeader]{\par
  \normalfont
  \topsep6\p@\@plus6\p@ \trivlist
  \item[\hskip\labelsep{\bfseries #1}\@addpunct{\bfseries.}]\ignorespaces
}{\endtrivlist}
\newcommand{\proofHeader}{証明}
\makeatother

% Sets.

\newcommand{\bbN}{\mathbb{N}}
\newcommand{\bbZ}{\mathbb{Z}}
\newcommand{\bbQ}{\mathbb{Q}}
\newcommand{\bbR}{\mathbb{R}}
\newcommand\set[1]{\{#1\}}
\newcommand\setc[2]{\{\,#1 \mid #2\,\}}
\newcommand\card[1]{\operatorname{card}(#1)}
\newcommand\xrange[2]{[#1, #2)}


\title{型システム入門(TaPL) 読書ノート}
\author{vain0}
\date{\today}
\begin{document}

\maketitle

\frontmatter

\addcontentsline{toc}{chapter}{概要}

『型システム入門 プログラミング言語と型の理論』の読書ノート。証明の細部などの行間や、演習の回答などを記す。

\tableofcontents

\mainmatter

\chapter{はじめに}

\chapter{数学的準備}

\begin{jexercise*}[2.2.6]
  略
\end{jexercise*}
\begin{jproof}
  $R$ を含む任意の反射的関係 $S$ をとる。
  任意の $(s, t) \in R'$ をとる。
  和集合の定義から、
    (a) $(s, t) \in R$ または
    (b) $(s, t) \in \setc{(s, s)}{s \in S}$
    である。
  \begin{description}
    \item[(a) のとき]
      $R \subset S$ より $(s, t) \in S$ となる。
    \item[(b) のとき]
      $S$ は反射的であるから、 $(s, t) = (s, s) \in S$ となる。
  \end{description}
  よって $R' \subset S$ となる。
  したがって、$R'$ は $R$ を含む最小の反射的関係である。
  すなわち、$R'$ は $R$ の反射的閉包といえる。
  \qed
\end{jproof}

\begin{jexercise*}[2.2.7]
  略
\end{jexercise*}
\begin{jproof}
  $R$ を含む任意の推移的関係 $S$ をとる。
  $R^+ \subset S$ を示せば十分である。
  任意の $n \in \bbN$ をとる。
  数学的帰納法により、$R_n \subset S$ を示す。
  $n = 0$ のとき、$R_0 = R \subset S$ である。
  次に $n > 0$ とする。
  任意の $(s, u) \in R_n$ をとる。
  $R_n$ の定義より、$(s, t) \in R_{n - 1}$ かつ $(t, u) \in R_{n - 1}$ となる $t$ がある。
  帰納法の仮定より、$R_{n - 1} \subset S$ である。
  $S$ は推移的なので、$(s, u) \in S$ がいえる。
  \qed
\end{jproof}


% \appendix
% \include{appendixA}

\chapter*{謝辞}
\addcontentsline{toc}{chapter}{謝辞}

感謝

% \include{biblography}

\newpage
\printindex

\end{document}
